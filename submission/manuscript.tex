% Options for packages loaded elsewhere
\PassOptionsToPackage{unicode}{hyperref}
\PassOptionsToPackage{hyphens}{url}
%
\documentclass[
  11pt,
]{article}
\usepackage{lmodern}
\usepackage{amssymb,amsmath}
\usepackage{ifxetex,ifluatex}
\ifnum 0\ifxetex 1\fi\ifluatex 1\fi=0 % if pdftex
  \usepackage[T1]{fontenc}
  \usepackage[utf8]{inputenc}
  \usepackage{textcomp} % provide euro and other symbols
\else % if luatex or xetex
  \usepackage{unicode-math}
  \defaultfontfeatures{Scale=MatchLowercase}
  \defaultfontfeatures[\rmfamily]{Ligatures=TeX,Scale=1}
\fi
% Use upquote if available, for straight quotes in verbatim environments
\IfFileExists{upquote.sty}{\usepackage{upquote}}{}
\IfFileExists{microtype.sty}{% use microtype if available
  \usepackage[]{microtype}
  \UseMicrotypeSet[protrusion]{basicmath} % disable protrusion for tt fonts
}{}
\makeatletter
\@ifundefined{KOMAClassName}{% if non-KOMA class
  \IfFileExists{parskip.sty}{%
    \usepackage{parskip}
  }{% else
    \setlength{\parindent}{0pt}
    \setlength{\parskip}{6pt plus 2pt minus 1pt}}
}{% if KOMA class
  \KOMAoptions{parskip=half}}
\makeatother
\usepackage{xcolor}
\IfFileExists{xurl.sty}{\usepackage{xurl}}{} % add URL line breaks if available
\IfFileExists{bookmark.sty}{\usepackage{bookmark}}{\usepackage{hyperref}}
\hypersetup{
  hidelinks,
  pdfcreator={LaTeX via pandoc}}
\urlstyle{same} % disable monospaced font for URLs
\usepackage[margin=1.0in]{geometry}
\usepackage{graphicx,grffile}
\makeatletter
\def\maxwidth{\ifdim\Gin@nat@width>\linewidth\linewidth\else\Gin@nat@width\fi}
\def\maxheight{\ifdim\Gin@nat@height>\textheight\textheight\else\Gin@nat@height\fi}
\makeatother
% Scale images if necessary, so that they will not overflow the page
% margins by default, and it is still possible to overwrite the defaults
% using explicit options in \includegraphics[width, height, ...]{}
\setkeys{Gin}{width=\maxwidth,height=\maxheight,keepaspectratio}
% Set default figure placement to htbp
\makeatletter
\def\fps@figure{htbp}
\makeatother
\setlength{\emergencystretch}{3em} % prevent overfull lines
\providecommand{\tightlist}{%
  \setlength{\itemsep}{0pt}\setlength{\parskip}{0pt}}
\setcounter{secnumdepth}{-\maxdimen} % remove section numbering
\usepackage{helvet} % Helvetica font
\renewcommand*\familydefault{\sfdefault} % Use the sans serif version of the font

%\usepackage{mathpazo} % Palatino font
%\renewcommand*\familydefault{\rmdefault} % Use the roman version of the font

\usepackage[T1]{fontenc}

\usepackage[none]{hyphenat}

\usepackage{setspace}
\doublespacing
\setlength{\parskip}{1em}

\usepackage{lineno}

\usepackage{pdfpages}

\author{}
\date{\vspace{-2.5em}}

\begin{document}

\vspace{10mm}

\hypertarget{ten-simple-rules-for-running-a-code-club-to-increasing-computational-skills-among-biologists}{%
\section{Ten Simple Rules for running a Code Club to increasing
computational skills among
biologists}\label{ten-simple-rules-for-running-a-code-club-to-increasing-computational-skills-among-biologists}}

\vspace{35mm}

Patrick D. Schloss\({^1}\)\({^\dagger}\)

\vspace{40mm}

\(\dagger\) To whom correspondence should be addressed:
\href{mailto:pschloss@umich.edu}{pschloss@umich.edu}

\(1\) Department of Microbiology and Immunology, University of Michigan,
Ann Arbor, MI 48109

\vspace{35mm}

\hypertarget{ten-simple-rules}{%
\subsubsection{Ten Simple Rules}\label{ten-simple-rules}}

\newpage
\linenumbers

\hypertarget{introduction}{%
\subsection{Introduction}\label{introduction}}

\begin{itemize}
\tightlist
\item
  We are awash in data
\item
  Traditional undergraduate and graduate training is insufficient to
  develop data analysis skills needed to analyze one's own data
\item
  Common to find exceptional bench scientists who are inexperienced at
  analyzing large datasets
\item
  As new people join a research group there is a mix of coding
  experience in the group
\end{itemize}

\hypertarget{our-solution}{%
\subsubsection{Our solution}\label{our-solution}}

\begin{itemize}
\tightlist
\item
  This is not unlike the situation researchers find themselves in
  regards to an ever growing literature
\item
  Developed a Code Club to parallel the traditional Journal Club
\item
  Have experimented with different strategies
\item
  Overall goal is to enable intra-research group learning surrounding
  topics in data analysis
\end{itemize}

\hypertarget{initial-format}{%
\subsubsection{Initial format}\label{initial-format}}

\begin{itemize}
\tightlist
\item
  Code review - project someone's script, go line by line through the
  code with each person in room stating the logic behind the code
\item
  Pros

  \begin{itemize}
  \tightlist
  \item
    Emphasizes importance of readability
  \item
    People get to see real, smelly code
  \end{itemize}
\item
  Cons

  \begin{itemize}
  \tightlist
  \item
    Can be hard to integrate logic across lines of code
  \item
    Although beginners could contribute, experts dominated conversation
  \item
    Could become adversarial, especially if the presenter didn't
    articulate their goal for the review
  \item
    Not easy for presenter to take information back to modify their own
    code
  \end{itemize}
\end{itemize}

\hypertarget{current-format}{%
\subsubsection{Current format}\label{current-format}}

\begin{enumerate}
\def\labelenumi{\arabic{enumi}.}
\tightlist
\item
  Help!

  \begin{itemize}
  \tightlist
  \item
    Clearly state problem the presenter would like to solve
  \item
    Distribute code to pairs of lab members
  \item
    Pairs work on problem
  \item
    Diversity of solutions shared back to group and presenter
  \item
    Ex: make code DRY, convert to tidyverse framework, go through issue
    tracker fixing code
  \end{itemize}
\item
  Tutorial

  \begin{itemize}
  \tightlist
  \item
    Introduce new package or way of doing things
  \item
    Start with brief presentation of package, workflow, book chapter
  \item
    Break into pairs to work through exercises
  \item
    Diversity of solutions shared back to group
  \item
    Ex: base vs ggplot, implement make/snakemake workflows
  \end{itemize}
\end{enumerate}

\hypertarget{rule-1-reciprocate-respect-group}{%
\subsection{Rule 1: Reciprocate respect
(Group)}\label{rule-1-reciprocate-respect-group}}

\begin{enumerate}
\def\labelenumi{\arabic{enumi}.}
\tightlist
\item
  Sign up
\end{enumerate}

\begin{itemize}
\tightlist
\item
  Establish a code of conduct
\end{itemize}

\begin{enumerate}
\def\labelenumi{\arabic{enumi}.}
\setcounter{enumi}{3}
\tightlist
\item
  Check your attitude

  \begin{enumerate}
  \def\labelenumii{\alph{enumii}.}
  \tightlist
  \item
    Be humble
  \item
    Be willing to incorporate feedback
  \item
    Ask clarifying questions
  \item
    Remember that this isn't a competition
  \end{enumerate}
\item
  Reciprocate respect

  \begin{enumerate}
  \def\labelenumii{\alph{enumii}.}
  \tightlist
  \item
    Facilitation

    \begin{enumerate}
    \def\labelenumiii{\roman{enumiii}.}
    \tightlist
    \item
      Lead the discussion
    \item
      Have an action plan: Focus the conversation
    \item
      Balance expertise in groups
    \item
      Attempt to purposefully engage early learners
    \item
      Avoid letting (senior) individuals dominate
    \end{enumerate}
  \item
    Attitude:

    \begin{enumerate}
    \def\labelenumiii{\roman{enumiii}.}
    \tightlist
    \item
      Both:

      \begin{enumerate}
      \def\labelenumiv{\arabic{enumiv}.}
      \tightlist
      \item
        Be humble
      \item
        Be willing to incorporate feedback
      \item
        Ask clarifying questions and note why you need the clarification
      \end{enumerate}
    \item
      Participants:

      \begin{enumerate}
      \def\labelenumiv{\arabic{enumiv}.}
      \tightlist
      \item
        Remember that this isn't a competition
      \item
        Focus on the speaker's goals
      \item
        Avoid distractions and engage with the presenter
      \item
        Allow others to contribute
      \item
        Give feedback \& coding suggestions at a novice explanation
        level
      \item
        Give compliments
      \end{enumerate}
    \end{enumerate}
  \end{enumerate}
\end{enumerate}

\hypertarget{rule-2-set-specific-goals-presenter}{%
\subsection{Rule 2: Set specific goals
(Presenter)}\label{rule-2-set-specific-goals-presenter}}

Preparation 5. Lead the discussion a. Have an action plan c.~Focus the
conversation 1. Set specific goals for each code review - Try to be as
explicit and clear as possible for what you are asking the participants
to do a. I have tried to be (what felt to me) painfully clear but I
still was falling short making it clear to some people. b. People will
always tend to notice or ask about things that are not the focus of the
exercise, being specific about the goal may help. Also setting up a
simplified scenario can be helpful (although requires much more prep
time)

\hypertarget{rule-3-keep-it-simple-presenter}{%
\subsection{Rule 3: Keep it simple
(Presenter)}\label{rule-3-keep-it-simple-presenter}}

\begin{enumerate}
\def\labelenumi{\arabic{enumi}.}
\setcounter{enumi}{1}
\tightlist
\item
  Practice exercise

  \begin{enumerate}
  \def\labelenumii{\alph{enumii}.}
  \tightlist
  \item
    When in doubt try it out! Run it by someone prior to lab meeting to
    make sure your request/goal is understandable
  \item
    Running on different systems will inevitably run into issues. Send
    out basic script ahead of time to try the dependencies/functions of
    focus
  \end{enumerate}
\item
  Create a \emph{minimal working example} - Working directly with your
  own raw code can be challenging

  \begin{enumerate}
  \def\labelenumii{\alph{enumii}.}
  \tightlist
  \item
    It is likely filled with code specific to you/your project
  \item
    It can take a lot of time to explain details and working of code
  \item
    Code can have multiple steps that are needed to run to get to script
    focus
  \item
    Using a whole script opens up the risk of getting off topic
  \end{enumerate}
\item
  Keep It Simple Silly (KISS)

  \begin{enumerate}
  \def\labelenumii{\alph{enumii}.}
  \tightlist
  \item
    There is never enough time

    \begin{enumerate}
    \def\labelenumiii{\roman{enumiii}.}
    \tightlist
    \item
      I typically have multiple activities planned, but rarely get
      through them all
    \item
      Modularize the activity so that each can be its own and if
      accomplished you are able to move onto the next
    \end{enumerate}
  \item
    One activity per code review is typically sufficient

    \begin{enumerate}
    \def\labelenumiii{\roman{enumiii}.}
    \tightlist
    \item
      This will take up most of the time

      \begin{enumerate}
      \def\labelenumiv{\arabic{enumiv}.}
      \tightlist
      \item
        Intro - 10 min
      \item
        Think/pair - 30 min
      \item
        Share in group - 10 min
      \item
        Debrief - 5 min
      \end{enumerate}
    \item
      I have not been successful in trying to do more than that in
      different ways because it is going to be taking time away from one
      of the other steps
    \end{enumerate}
  \end{enumerate}
\end{enumerate}

\hypertarget{rule-4-give-group-time-to-prepare-presenter}{%
\subsection{Rule 4: Give group time to prepare
(Presenter)}\label{rule-4-give-group-time-to-prepare-presenter}}

\begin{enumerate}
\def\labelenumi{\arabic{enumi}.}
\setcounter{enumi}{2}
\tightlist
\item
  Communicate with the group in advance so they can prepare

  \begin{enumerate}
  \def\labelenumii{\alph{enumii}.}
  \tightlist
  \item
    Provide code chunk and necessary data
  \item
    Provide instructions for appropriate dependencies
  \item
    Provide use case (where available/appropriate)
  \item
    Specify goals \& expectations
  \end{enumerate}
\end{enumerate}

\hypertarget{rule-5-prepare-in-advance-audience}{%
\subsection{Rule 5: Prepare in advance
(Audience)}\label{rule-5-prepare-in-advance-audience}}

\begin{enumerate}
\def\labelenumi{\arabic{enumi}.}
\tightlist
\item
  Prepare

  \begin{enumerate}
  \def\labelenumii{\alph{enumii}.}
  \tightlist
  \item
    Review the code in advance
  \item
    Download datasets, install packages
  \item
    Google what you don't understand
  \end{enumerate}
\end{enumerate}

Read chapter, paper, etc.

\hypertarget{rule-6-dont-give-busy-work-presenter}{%
\subsection{Rule 6: Don't give busy work
(Presenter)}\label{rule-6-dont-give-busy-work-presenter}}

\begin{enumerate}
\def\labelenumi{\arabic{enumi}.}
\setcounter{enumi}{1}
\tightlist
\item
  For Help! format, select your code

  \begin{enumerate}
  \def\labelenumii{\alph{enumii}.}
  \tightlist
  \item
    That you are willing to receive and incorporate feedback on
  \item
    That is digestible in an hour (\textless100 lines)
  \item
    Use ``vanilla'' R to check for dependencies on a ``fresh''
    environment
  \item
    Prepare a small example dataset
  \end{enumerate}
\end{enumerate}

\hypertarget{rule-7-include-all-levels-of-learner-presenter}{%
\subsection{Rule 7: Include all levels of learner
(Presenter)}\label{rule-7-include-all-levels-of-learner-presenter}}

\begin{itemize}
\tightlist
\item
  Ask questions that are appropriate for different levels
\item
  Be purposeful about how you pair people off

  \begin{itemize}
  \tightlist
  \item
    Use think/pair/share
  \item
    Senior/Beginner
  \item
    Senior/senior and Beginner/beginner
  \item
    Self-selected
  \item
    Navigator/driver with a switch midway through
  \end{itemize}
\end{itemize}

\hypertarget{rule-8-engage-with-the-material-audience}{%
\subsection{Rule 8: Engage with the material
(Audience)}\label{rule-8-engage-with-the-material-audience}}

\begin{enumerate}
\def\labelenumi{\arabic{enumi}.}
\setcounter{enumi}{2}
\tightlist
\item
  Participate

  \begin{enumerate}
  \def\labelenumii{\alph{enumii}.}
  \tightlist
  \item
    Focus on the speaker's goals
  \item
    Avoid distractions and engage with the presenter
  \item
    Allow others to contribute
  \item
    Ask clarifying questions and note why you need the clarification
  \item
    Give feedback \& coding suggestions at a novice explanation level
  \item
    Give compliments
  \end{enumerate}
\end{enumerate}

\hypertarget{rule-9-let-the-material-change-you-audience}{%
\subsection{Rule 9: Let the material change you
(Audience)}\label{rule-9-let-the-material-change-you-audience}}

\begin{itemize}
\tightlist
\item
  Reflection

  \begin{itemize}
  \tightlist
  \item
    Use what you've learned to refactor your own code / improve new code
  \end{itemize}
\end{itemize}

\hypertarget{rule-10-speak-up-group}{%
\subsection{Rule 10: Speak up (Group)}\label{rule-10-speak-up-group}}

\begin{itemize}
\tightlist
\item
  Speak up!

  \begin{itemize}
  \tightlist
  \item
    Topics - check interest level of potential topics - check group is
    at sufficient skill level needed - need for specific skills/methods
    in group - May need to circle back to ``beginner'' stuff with
    natural cycle of research group development

    \begin{itemize}
    \tightlist
    \item
      What works and doesn't
    \end{itemize}
  \end{itemize}
\end{itemize}

\hypertarget{conclusion}{%
\subsection{Conclusion}\label{conclusion}}

\hypertarget{acknowledgements}{%
\subsection{Acknowledgements}\label{acknowledgements}}

\newpage

\hypertarget{references}{%
\subsection{References}\label{references}}

\hypertarget{refs}{}

\url{https://journals.plos.org/ploscompbiol/article?id=10.1371/journal.pcbi.1004526}

\newpage

\textbf{Figure 1. XXXXX XXX XXXXX XXX XXXXX XXX XXXXX XXX XXXXX XXX
XXXXX XXX.} XXXXX XXX XXXXX XXX XXXXX XXX XXXXX XXX XXXXX XXX XXXXX XXX
XXXXX XXX XXXXX XXX XXXXX XXX XXXXX XXX XXXXX XXX XXXXX XXX XXXXX XXX
XXXXX XXX XXXXX XXX XXXXX XXX.

\end{document}
