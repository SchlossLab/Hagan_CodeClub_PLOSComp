% Options for packages loaded elsewhere
\PassOptionsToPackage{unicode}{hyperref}
\PassOptionsToPackage{hyphens}{url}
%
\documentclass[
  11pt,
]{article}
\usepackage{lmodern}
\usepackage{amssymb,amsmath}
\usepackage{ifxetex,ifluatex}
\ifnum 0\ifxetex 1\fi\ifluatex 1\fi=0 % if pdftex
  \usepackage[T1]{fontenc}
  \usepackage[utf8]{inputenc}
  \usepackage{textcomp} % provide euro and other symbols
\else % if luatex or xetex
  \usepackage{unicode-math}
  \defaultfontfeatures{Scale=MatchLowercase}
  \defaultfontfeatures[\rmfamily]{Ligatures=TeX,Scale=1}
\fi
% Use upquote if available, for straight quotes in verbatim environments
\IfFileExists{upquote.sty}{\usepackage{upquote}}{}
\IfFileExists{microtype.sty}{% use microtype if available
  \usepackage[]{microtype}
  \UseMicrotypeSet[protrusion]{basicmath} % disable protrusion for tt fonts
}{}
\makeatletter
\@ifundefined{KOMAClassName}{% if non-KOMA class
  \IfFileExists{parskip.sty}{%
    \usepackage{parskip}
  }{% else
    \setlength{\parindent}{0pt}
    \setlength{\parskip}{6pt plus 2pt minus 1pt}}
}{% if KOMA class
  \KOMAoptions{parskip=half}}
\makeatother
\usepackage{xcolor}
\IfFileExists{xurl.sty}{\usepackage{xurl}}{} % add URL line breaks if available
\IfFileExists{bookmark.sty}{\usepackage{bookmark}}{\usepackage{hyperref}}
\hypersetup{
  hidelinks,
  pdfcreator={LaTeX via pandoc}}
\urlstyle{same} % disable monospaced font for URLs
\usepackage[margin=1.0in]{geometry}
\usepackage{longtable,booktabs}
% Correct order of tables after \paragraph or \subparagraph
\usepackage{etoolbox}
\makeatletter
\patchcmd\longtable{\par}{\if@noskipsec\mbox{}\fi\par}{}{}
\makeatother
% Allow footnotes in longtable head/foot
\IfFileExists{footnotehyper.sty}{\usepackage{footnotehyper}}{\usepackage{footnote}}
\makesavenoteenv{longtable}
\usepackage{graphicx,grffile}
\makeatletter
\def\maxwidth{\ifdim\Gin@nat@width>\linewidth\linewidth\else\Gin@nat@width\fi}
\def\maxheight{\ifdim\Gin@nat@height>\textheight\textheight\else\Gin@nat@height\fi}
\makeatother
% Scale images if necessary, so that they will not overflow the page
% margins by default, and it is still possible to overwrite the defaults
% using explicit options in \includegraphics[width, height, ...]{}
\setkeys{Gin}{width=\maxwidth,height=\maxheight,keepaspectratio}
% Set default figure placement to htbp
\makeatletter
\def\fps@figure{htbp}
\makeatother
\setlength{\emergencystretch}{3em} % prevent overfull lines
\providecommand{\tightlist}{%
  \setlength{\itemsep}{0pt}\setlength{\parskip}{0pt}}
\setcounter{secnumdepth}{-\maxdimen} % remove section numbering
\usepackage{helvet} % Helvetica font
\renewcommand*\familydefault{\sfdefault} % Use the sans serif version of the font

%\usepackage{mathpazo} % Palatino font
%\renewcommand*\familydefault{\rmdefault} % Use the roman version of the font

\usepackage[T1]{fontenc}

\usepackage[none]{hyphenat}

\usepackage{setspace}
\doublespacing
\setlength{\parskip}{1em}

\usepackage{lineno}

\usepackage{pdfpages}

\author{}
\date{\vspace{-2.5em}}

\begin{document}

\vspace{10mm}

\hypertarget{ten-simple-rules-for-running-a-code-club-for-increasing-computational-skills-among-biologists}{%
\section{Ten Simple Rules for running a Code Club for increasing
computational skills among
biologists}\label{ten-simple-rules-for-running-a-code-club-for-increasing-computational-skills-among-biologists}}

\vspace{35mm}

\emph{Author names here, order TBD}

Patrick D. Schloss\({^1}\)\({^\dagger}\)

\vspace{40mm}

\(\dagger\) To whom correspondence should be addressed:
\href{mailto:pschloss@umich.edu}{pschloss@umich.edu}

\(1\) Department of Microbiology and Immunology, University of Michigan,
Ann Arbor, MI 48109

\vspace{35mm}

\hypertarget{ten-simple-rules}{%
\subsubsection{Ten Simple Rules}\label{ten-simple-rules}}

\newpage
\linenumbers

\hypertarget{introduction}{%
\subsection{Introduction}\label{introduction}}

For most biologists, the ability to generate data has outpaced the
ability to analyze those data. High throughput data comes to us from DNA
and RNA sequencing, flow cytometry, metabolomics, molecular screens, and
more. Although some accept the approach of compartmentalizing the
ability to generate data and analyze data, we have found scientists feel
empowered when they have the ability to both ask and answer their own
biological questions. Yet, the standard undergraduate and graduate
training in the biological sciences is insufficient to develop the data
analysis skills needed to analyze one's data. In our experience
performing microbiome research, it is more common to find exceptional
bench scientists who are inexperienced at analyzing large datasets than
to find the reverse. Of course this raises a challenge: How do we train
bench scientists to analyze datasets to answer biological questions?

The ever-growing ability to generate data and constant feeling of
helplessness in analyzing it is analogous to the struggles we also face
with engaging the voluminous scientific literature. Within our research
group we have leveraged this similarity to address the challenges of
training bench scientists to engage in reproducible data analysis
practices. Our group and others hold a regular Journal Club to stay
current on the literature and model to junior scientists how to engage
that literature {[}1{]}. Over the past 4 years we have experimented with
using a Code Club to introduce new software packages, solve specific
problems, and evaluate each other's code.

Initially, the Code Club was created as a way to review code that was
being used in different projects. We would project someone's script onto
a screen and go line by line through the code with each person in the
room stating the logic behind the code. This approach emphasized the
importance of readability and it gave beginners the opportunity to see
the real, ``smelly'' code of more experienced people in the group.
Unfortunately, the format only allowed us to review a small fraction of
a project's code base and it became difficult to integrate the
programmer's logic across their full project. We quickly realized that
although beginners could contribute to the discussion, the more senior
group members dominated conversation. This frequently created an
adversarial environment, which led to the presenter not offering to
present again. From this experience we learned that it is critical for
the presenter to clearly articulate their goals for the Code Club and
that the participants needed to engage with the code in that context.

Our Code Club sessions are each an hour long and are held roughly every
other week as part of a weekly two hour lab meeting. Presenters
voluntarily sign up to lead a session. The current format of our Code
Club generally takes one of two approaches. The first is a more humane
version of a group code review. The presenter will clearly state the
problem they would like to solve, break the participants into smaller
groups, and then ask each group to solve the problem. For example,
someone may have an R script where a chunk of code is repeated
throughout the script. The challenge for the session would be to convert
the chunk into a function and call it throughout the script to make it
``DRY'' (i.e.~Don't Repeat Yourself){[}2{]}. The presenter leaves the
session with several partial or working solutions to their problem and
everyone relearns the importance of writing DRY code. The second is a
tutorial approach. The presenter introduces a new package or way of
doing things. They then give the group an activity that forces them to
replace their current approach with a new approach. For example, one of
the more memorable activities was giving participants raw data and a
finished plot and asking two pairs of participants to generate the plot
using base R and asking the other two pairs to generate the plot with
ggplot. The stipulation was that base R users had to build the plots in
ggplot and ggplot users had to build the plot in base R. Each Code Club
ends with an opportunity to report back to the presenter and larger
group. Our Code Clubs typically have 7 to 10 participants. Because of
the ``think-pair-share'' approach that is baked into each session it
could likely be scaled to larger groups {[}3{]}.

We continue to experiment with approaches to running Code Club. Although
any individual Code Club session may be be more experimental than
others, on the whole they have been a critical tool to providing the
much needed training to use reproducible practices for analyzing data.
Examples of successful code club topics are described in Table 1. These
Ten Rules summarize what we have learned as presenters and participants.

\hypertarget{rule-1-reciprocate-respect}{%
\subsection{Rule 1: Reciprocate
respect}\label{rule-1-reciprocate-respect}}

It is critical that the presenter and participants respect each other
and that there be an individual (e.g.~the lab director) that enforces a
code of conduct. Everyone in the session needs to have the humility to
acknowledge that they still have more to learn about any topic. We have
found that many problems are avoided when the presenter takes charge of
the session by having a clear lesson plan, thoughtfully creates groups,
and gives encouragement. Similarly, participants foster a positive
environment by remembering that the task is not a competition, focusing
on the presenter's goals, allowing their partner to contribute, asking
clarifying questions when appropriate, and avoiding distractions
(e.g.~email, social media). Learning to program is challenging and too
often it can lead to a toxic environment. All parties in a Code Club are
responsible for preventing this toxicity.

\hypertarget{rule-2-set-specific-goals}{%
\subsection{Rule 2: Set specific
goals}\label{rule-2-set-specific-goals}}

In our early Code Clubs we noticed that if the presenter did not clearly
state their goals for the session, it would often led to frustration for
the presenter and participants. If the presenter shared their own code,
did they want us to focus on their coding style or do they want help
with incorporating a new package into their workflow? Participants will
always notice or ask about things that are not the focus of the
exercise. Be specific about the goal and bring them back to the task if
they are distracted. Where possible, create a simplified scenario
(i.e.~minimal, reproducible example {[}4{]}) can be helpful that removes
distractions. The presenter should make sure that the simplified example
works and behaves the way they expect before coming to the Code Club.
Beyond the content, participants should know what they can be expected
to have completed by the end of the session.

\hypertarget{rule-3-keep-it-simple}{%
\subsection{Rule 3: Keep it simple}\label{rule-3-keep-it-simple}}

Our Code Club needs to fit within an hour slot. The presenter should
anticipate enough time for an introduction and brief instruction, time
for participants to engage the material, and time for everyone to report
back. A typical schedule for Code Review is 10 minutes of introduction
and instruction, 30 minutes of paired programming, 5 minutes to get
groups to wrap up, and 10 minutes to report back to the group. Remember
that a task that you can accomplish in 10 minutes may take a more junior
member of your group 30 minutes. The goal of the Code Club should be to
help that junior member to learn the topic and to reinforce the topic
for the more senior person and help them see the concept in a new
context. One possibility would be to create ``stretch goals'' for those
that complete the task more quickly than everyone else. We once had a
presenter try to teach the group some basic Julia syntax. By the time
people installed the interpreter, the time was up. Alternatively, a
session where participants were asked to make code DRY gave them
multiple stopping points where they could feel successful
(e.g.~commenting code, creating function, implementing function,
refactoring function).

\hypertarget{rule-4-give-participants-time-to-prepare}{%
\subsection{Rule 4: Give participants time to
prepare}\label{rule-4-give-participants-time-to-prepare}}

Similar to a Journal Club, the presenter should give participants a few
days to prepare for the Code Club. Be sure to tell the participants what
you will be covering. Provide the them with instructions on how to
install dependencies, download data, and get the initial code. Make sure
that the materials work together as intended. Considering the compressed
schedule, asking participants to download materials beforehand is
helpful. We have found that using separate GitHub repositories for each
session can be a helpful way to get information to participants. Of
course, using git and GitHub to engage in collaborative coding should
probably be its own Code Club before using the tools more broadly with
the group. An easier entry point is to have groups post their code to a
Slack channel that we use to announce materials for our meetings.

\hypertarget{rule-5-prepare-in-advance-to-maximize-your-participation}{%
\subsection{Rule 5: Prepare in advance to maximize your
participation}\label{rule-5-prepare-in-advance-to-maximize-your-participation}}

It is the rare scientist that can fully participate in a Journal Club
discussion about a paper that they have not read. Coming to Code Club
without installing a dependency is like going to Journal Club and asking
a simple question about the paper. It shows a lack of respect and
preparation. Following Rule 4, review the code in advance, download the
datasets, install the necessary packages, and perhaps read up on the
topic. If your Code Club is reading a paper or chapters in a data
science book be sure to read them before the session (e.g.~{[}5--7{]}).
Consider how you could use the material for your own work.

\hypertarget{rule-6-dont-give-your-participants-busy-work}{%
\subsection{Rule 6: Don't give your participants busy
work}\label{rule-6-dont-give-your-participants-busy-work}}

Participants want to learn something because it will be useful to them
or because they think it will help the presenter to improve their code.
Presenters should do their best to satisfy those desires. It would not
make much sense to hold a Code Club on downloading stock market data if
the group was not interested in those data. At the same time,
participants should not be asked to help the presenter improve their
code if the presenter has no intention of incorporating the suggestions.
It would be useful to keep a list of packages or tasks that members of
the group want to learn about. That way if a presenter is having a hard
time coming up with ideas, they can take something from the list and
present it.

\hypertarget{rule-7-include-all-levels-of-participants}{%
\subsection{Rule 7: Include all levels of
participants}\label{rule-7-include-all-levels-of-participants}}

A significant challenge to presenting at Code Club is to pick topics and
activities that appeal to a critical mass of the participants. For
example, if there has recently been a large turn over in group
membership, it can be useful to go back to the basics and cover commonly
used tools. We have identified several strategies to overcome the
challenges of having varying skill levels. As an alternative to letting
participants form their own pairs, the presenter can be purposeful in
how they form pairs. Depending on the goals of the session, forming
pairs between participants that have similar or different skills can be
effective. If pairs are formed where partners have similar skill levels,
then the presenter needs to design activities that are appropriate to
different levels of participants. We have found that a good activity for
beginners is to have them comment code, which forces them to dissect a
code chunk and understand what each line does. An advantage of forming
pairs with disparate skill levels is that all of the pairs can be given
the same task leading to a diversity of methods to achieve the same
result. This approach to pairing also helps graft new members into the
group that have emerging programming skills. Regardless of how partners
are selected, asking the pairs to identify a navigator and driver is
helpful {[}8{]}. The driver types at the computer while the navigator
tells them what to type. Midway through the activity the presenter can
tell the partners to switch roles. Intentionally forming pairs can also
be a mechanism to engineer interactions by breaking up cliques, avoiding
potentially toxic combinations of members, or pairing reliable role
models with new members of the group.

\hypertarget{rule-8-participate}{%
\subsection{Rule 8: Participate}\label{rule-8-participate}}

Having an open laptop on the table and permission to use it can feel
like an invitation to get distracted by other work, emails, and browsing
the internet. Fight that urge and focus on the presenter's goals. Allow
your partner to contribute. Speak up for yourself and force your partner
to let you contribute. Even if the material seems too advanced for you,
programming languages like R and Python are generally expressive and
should allow you to engage with the logic, even if the syntax is too
hard. Frankly, understanding the logic of when to use one modeling
approach over another is more important than how to use the modeling
approach. If you understand the ``why'', the ``how'' will generally
quickly follow. If you are a more experienced participant, give your
feedback and coding suggestions at a level that seeks to inform
everyone. Regardless of your skill level, your partner and the presenter
are putting themselves at risk by revealing what they do or do not know.
Encourage them and show your gratitude for helping you learn something
new.

\hypertarget{rule-9-let-the-material-change-you}{%
\subsection{Rule 9: Let the material change
you}\label{rule-9-let-the-material-change-you}}

Part of the humility required to participate in a Code Club is both
acknowledging your incomplete training and that learning something new
is possible. Assume that the presenter has followed the other rules and
that they think you should be using what they presented. After the Code
Club, try to incorporate that material into your new code or by
refactoring your old code. By practicing the material in a different
context you will learn the material better. You may identify concepts
that you are unsure of, which gives you another opportunity to ask the
presenter for help.

\hypertarget{rule-10-experiment}{%
\subsection{Rule 10: Experiment!}\label{rule-10-experiment}}

The selection of content and structure for each Code Club is determined
by the presenter making the process democratic and distributed. If
someone thinks something is worth learning and wants to teach it, they
have that power. If they want to experiment with a different format,
they are free to try it out. Members of the Code Club need to feel like
they have the power to shape the direction of the group. If members are
following these Rules, they will naturally reflect on the skills and
interests of the other members in the group. For example, there is
always turn over in a research group and thus it is important to make
sure that there are opportunities to go back to things people consider
``simple'' to teach the content to new people or provide a refresher to
people that perhaps have been at the lab bench more than at the computer
recently. The group should also feel free to experiment with the format.
If we try a different format, then it is useful to give everyone an
opportunity to debrief on what they did or did not like with the new
approach.

\hypertarget{conclusion}{%
\subsection{Conclusion}\label{conclusion}}

The most important rules are the first and last. Members of the Code
Club need to feel comfortable with other members of the group and
sufficiently empowered to try something new or to ask for help. Aside
from growing in our programming skills, we have noticed two other
benefits that point to the efforts of creating a positive culture.
First, we have intentionally interviewed postdoc candidates on the days
we have held Code Club. Although it is important to make it clear that
they are not being assessed on their coding skills, it is useful to see
how the candidate interacts with other members of the research group in
a relatively low stakes environment. At the same time, the candidate can
learn about the culture of the research group. Second, members of other
research groups where they are isolated in their desire to grow in these
skills have integrated themselves into our Code Club. This speaks to the
broader need for Code Club and the likelihood that it would work when
expanded to a larger group of individuals that are not necessarily in
the same research group. Ultimately, our Code Club has helped
researchers feel more comfortable seeking out their colleagues outside
of Code Club sessions for help in improving their code.

\hypertarget{acknowledgements}{%
\subsection{Acknowledgements}\label{acknowledgements}}

All co-authors have been participants in the Code Clubs run as part of
PDS's lab meetings. They have had a critical role in shaping the
evolution of the sessions. The order of co-authors was determined
by\ldots{}

\newpage

\hypertarget{references}{%
\subsection{References}\label{references}}

\hypertarget{refs}{}
\leavevmode\hypertarget{ref-Lonsdale2016}{}%
1. Lonsdale A, Penington JS, Rice T, Walker M, Dashnow H. Ten simple
rules for a bioinformatics journal club. Lewitter F, editor. PLOS
Computational Biology. 2016;12: e1004526.
doi:\href{https://doi.org/10.1371/journal.pcbi.1004526}{10.1371/journal.pcbi.1004526}

\leavevmode\hypertarget{ref-Wilson2014}{}%
2. Wilson G, Aruliah DA, Brown CT, Hong NPC, Davis M, Guy RT, et al.
Best practices for scientific computing. Eisen JA, editor. PLoS Biology.
2014;12: e1001745.
doi:\href{https://doi.org/10.1371/journal.pbio.1001745}{10.1371/journal.pbio.1001745}

\leavevmode\hypertarget{ref-Kothiyal2014}{}%
3. Kothiyal A, Murthy S, Iyer S. Think-pair-share in a large CS1 class.
Proceedings of the 2014 conference on innovation \& technology in
computer science education - ITiCSE 14. ACM Press; 2014.
doi:\href{https://doi.org/10.1145/2591708.2591739}{10.1145/2591708.2591739}

\leavevmode\hypertarget{ref-StackOverflow_MREE}{}%
4. Stack Overflow. How to create a minimal, reproducible example. 2020.
Available:
\url{https://stackoverflow.com/help/minimal-reproducible-example}

\leavevmode\hypertarget{ref-Wickham_2016}{}%
5. Wickham H, Grolemund G. O'Reilly Media; 2016. Available:
\url{https://r4ds.had.co.nz}

\leavevmode\hypertarget{ref-Schloss_2018}{}%
6. Schloss PD. The riffomonas reproducible research tutorial series.
Journal of Open Source Education. 2018;1: 13.
doi:\href{https://doi.org/10.21105/jose.00013}{10.21105/jose.00013}

\leavevmode\hypertarget{ref-Noble2009}{}%
7. Noble WS. A quick guide to organizing computational biology projects.
Lewitter F, editor. PLoS Computational Biology. 2009;5: e1000424.
doi:\href{https://doi.org/10.1371/journal.pcbi.1000424}{10.1371/journal.pcbi.1000424}

\leavevmode\hypertarget{ref-Bockeler_2020}{}%
8. Böckeler B, Siessegger N. On pair programming. 2020. Available:
\url{https://martinfowler.com/articles/on-pair-programming.html}

\newpage

\textbf{Table 1. Examples of successful Code Club topics.}

\begin{longtable}[]{@{}ll@{}}
\toprule
\begin{minipage}[b]{0.25\columnwidth}\raggedright
\textbf{Title}\strut
\end{minipage} & \begin{minipage}[b]{0.69\columnwidth}\raggedright
\textbf{Description}\strut
\end{minipage}\tabularnewline
\midrule
\endhead
\begin{minipage}[t]{0.25\columnwidth}\raggedright
base vs.~ggplot\strut
\end{minipage} & \begin{minipage}[t]{0.69\columnwidth}\raggedright
Given input data and a figure, recreate the figure using R's base
graphics or ggplot syntax\strut
\end{minipage}\tabularnewline
\begin{minipage}[t]{0.25\columnwidth}\raggedright
Snakemake\strut
\end{minipage} & \begin{minipage}[t]{0.69\columnwidth}\raggedright
Given a bash script that contains an analysis pipeline, convert it to a
series of recipes (can also be done with GNU Make)\strut
\end{minipage}\tabularnewline
\begin{minipage}[t]{0.25\columnwidth}\raggedright
DRYing code\strut
\end{minipage} & \begin{minipage}[t]{0.69\columnwidth}\raggedright
Given script with repeated code, create functions to remove
repetition\strut
\end{minipage}\tabularnewline
\begin{minipage}[t]{0.25\columnwidth}\raggedright
mothur and Vegan\strut
\end{minipage} & \begin{minipage}[t]{0.69\columnwidth}\raggedright
Given a pairwise community dissimilarity matrix, compare communities
using the \texttt{adonis} function in the Vegan R package\strut
\end{minipage}\tabularnewline
\begin{minipage}[t]{0.25\columnwidth}\raggedright
tidy data\strut
\end{minipage} & \begin{minipage}[t]{0.69\columnwidth}\raggedright
Given a wide-formatted data table, convert it to a long, tidy-formatted
data table using tools from R's tidyverse\strut
\end{minipage}\tabularnewline
\begin{minipage}[t]{0.25\columnwidth}\raggedright
GitFlow\strut
\end{minipage} & \begin{minipage}[t]{0.69\columnwidth}\raggedright
Participants file and claim an issue to add their name to a README file
in a GitHub-hosted repository and file a pull request to complete the
issue\strut
\end{minipage}\tabularnewline
\begin{minipage}[t]{0.25\columnwidth}\raggedright
Integrating R and google docs\strut
\end{minipage} & \begin{minipage}[t]{0.69\columnwidth}\raggedright
Scrape a google docs workbook and clean the data to identify previous
Code Club presenters\strut
\end{minipage}\tabularnewline
\bottomrule
\end{longtable}

\end{document}
