\documentclass[11pt,]{article}
\usepackage{lmodern}
\usepackage{amssymb,amsmath}
\usepackage{ifxetex,ifluatex}
\usepackage{fixltx2e} % provides \textsubscript
\ifnum 0\ifxetex 1\fi\ifluatex 1\fi=0 % if pdftex
  \usepackage[T1]{fontenc}
  \usepackage[utf8]{inputenc}
\else % if luatex or xelatex
  \ifxetex
    \usepackage{mathspec}
  \else
    \usepackage{fontspec}
  \fi
  \defaultfontfeatures{Ligatures=TeX,Scale=MatchLowercase}
\fi
% use upquote if available, for straight quotes in verbatim environments
\IfFileExists{upquote.sty}{\usepackage{upquote}}{}
% use microtype if available
\IfFileExists{microtype.sty}{%
\usepackage[]{microtype}
\UseMicrotypeSet[protrusion]{basicmath} % disable protrusion for tt fonts
}{}
\PassOptionsToPackage{hyphens}{url} % url is loaded by hyperref
\usepackage[unicode=true]{hyperref}
\hypersetup{
            pdfborder={0 0 0},
            breaklinks=true}
\urlstyle{same}  % don't use monospace font for urls
\usepackage[margin=1.0in]{geometry}
\usepackage{longtable,booktabs}
% Fix footnotes in tables (requires footnote package)
\IfFileExists{footnote.sty}{\usepackage{footnote}\makesavenoteenv{long table}}{}
\usepackage{graphicx,grffile}
\makeatletter
\def\maxwidth{\ifdim\Gin@nat@width>\linewidth\linewidth\else\Gin@nat@width\fi}
\def\maxheight{\ifdim\Gin@nat@height>\textheight\textheight\else\Gin@nat@height\fi}
\makeatother
% Scale images if necessary, so that they will not overflow the page
% margins by default, and it is still possible to overwrite the defaults
% using explicit options in \includegraphics[width, height, ...]{}
\setkeys{Gin}{width=\maxwidth,height=\maxheight,keepaspectratio}
\IfFileExists{parskip.sty}{%
\usepackage{parskip}
}{% else
\setlength{\parindent}{0pt}
\setlength{\parskip}{6pt plus 2pt minus 1pt}
}
\setlength{\emergencystretch}{3em}  % prevent overfull lines
\providecommand{\tightlist}{%
  \setlength{\itemsep}{0pt}\setlength{\parskip}{0pt}}
\setcounter{secnumdepth}{0}
% Redefines (sub)paragraphs to behave more like sections
\ifx\paragraph\undefined\else
\let\oldparagraph\paragraph
\renewcommand{\paragraph}[1]{\oldparagraph{#1}\mbox{}}
\fi
\ifx\subparagraph\undefined\else
\let\oldsubparagraph\subparagraph
\renewcommand{\subparagraph}[1]{\oldsubparagraph{#1}\mbox{}}
\fi

% set default figure placement to htbp
\makeatletter
\def\fps@figure{htbp}
\makeatother

\usepackage{helvet} % Helvetica font
\renewcommand*\familydefault{\sfdefault} % Use the sans serif version of the font

%\usepackage{mathpazo} % Palatino font
%\renewcommand*\familydefault{\rmdefault} % Use the roman version of the font

\usepackage[T1]{fontenc}

\usepackage[none]{hyphenat}

\usepackage{setspace}
\doublespacing
\setlength{\parskip}{1em}

\usepackage{lineno}

\usepackage{pdfpages}

\author{}
\date{\vspace{-2.5em}}

\begin{document}

\vspace{10mm}

\section{Ten Simple Rules to increase computational skills among
biologists with Code
Clubs}\label{ten-simple-rules-to-increase-computational-skills-among-biologists-with-code-clubs}

\vspace{35mm}

Ada K. Hagan\({^2}\), Nick Lesniak, \emph{Author names here, order TBD}

Patrick D. Schloss\({^1}\)\({^\dagger}\)

\vspace{40mm}

\(\dagger\) To whom correspondence should be addressed:
\href{mailto:pschloss@umich.edu}{pschloss@umich.edu}

\(1\) Department of Microbiology and Immunology, University of Michigan,
Ann Arbor, MI 48109

\(2\) Alliance SciComm \& Consulting, Linden, MI 48418

\vspace{35mm}

\subsubsection{Ten Simple Rules}\label{ten-simple-rules}

\newpage

\linenumbers

\subsection{Introduction}\label{introduction}

For most biologists, the ability to generate data has outpaced the
ability to analyze those data. High throughput data comes to us from DNA
and RNA sequencing, flow cytometry, metabolomics, molecular screens, and
more. Although some accept the approach of compartmentalizing data
generation and data analysis, we have found scientists feel empowered
when they can both ask and answer their own biological questions. Yet,
the standard undergraduate and graduate training in the biological
sciences is insufficient for developing these advanced data analysis
skills. In our experience performing microbiome research, it is more
common to find exceptional bench scientists who are inexperienced at
analyzing large data sets than to find the reverse. Of course this
raises a challenge: How do we train bench scientists that analyze data
sets to answer biological questions?

The ever-growing ability to generate data and constant feeling of
helplessness in analyzing it is analogous to the struggles we also face
with engaging the voluminous scientific literature. A common strategy to
keeping up with the literature are journal clubs, which involve group
discussion of a pre-selected paper. These range from informal
discussions to PowerPoint presentations and course credit. In addition
to staying current on the literature, journal clubs help strengthen
skills in critical thinking, communication, and literature review
{[}1{]}. Our research group has leveraged the similarity between the
overwhelming natures of scientific literature and data analysis to
address the challenge of teaching reproducible data analysis practices
to bench scientists. Over the past 4 years we have experimented with a
Code Club model to improve data analysis skills in a community
environment.

Our Code Club sessions are an hour long and alternate with Journal Club
as the second part of weekly two hour lab meetings. Presenters volunteer
to lead a session with the expectation that they alternate leading
Journal and Code Clubs. Initially, the Code Club was used to review code
from trainee projects. Instead of a presentation, the presenter would
project their code onto a screen and the participants would go line by
line through the code, stating the logic behind each line. This approach
emphasized the importance of code readability and gave beginners the
opportunity to see the real-life, messy code of more experienced peers.
Unfortunately, the format only allowed us to review a fraction of a
project's code base, making it difficult to integrate the programmer's
logic across their full project. A major issue with this model was that
sometimes beginners couldn't contribute to improving the code, and even
when they could, more senior group members would eventually dominate the
discussion. This led to a lack of participation by beginners, who would
mentally check out to focus on other activities, and frequently resulted
in an adversarial environment between the presenter and senior group
members. As a result, presenters rarely offered to present Code Club
again.

From these experiences, we began a collective conversation to improve
our Code Club model and have evolved it into two approaches. The first
is a more constructive version of a group code review. The presenter
clearly states the problem they want to solve, breaks the participants
into smaller groups, and then asks each group to solve the problem, or a
portion of it. For example, someone may have an R script with a
repeating chunk of code. The challenge for the session would be to
convert the code chunk into a function to be called throughout the
script to make it ``DRY'' (i.e.~Don't Repeat Yourself){[}2{]}. The
presenter leaves the session with several partial or working solutions
to their problem and the importance of writing DRY code is reinforced.
The second approach is a tutorial. The presenter introduces a new
package or technique and assigns an activity to practice the new
approach. For example, a more memorable activity was giving participants
raw data and a finished plot. Paired participants were tasked with
generating the plot using either base R or ggplot, but base R users had
to use ggplot and vice versa. In either approach, the Code Club ends
with a report back to the presenter and larger group describing the
approach each pair took. Our Code Clubs typically have 7 to 10
participants, but the inherent ``think-pair-share'' strategy should
allow it to be scaled to larger groups {[}3{]}.

We continue to experiment with approaches to running Code Club, but have
learned that it is critical for the presenter to clearly articulate
their goals and facilitate participant engagement. Although any
individual Code Club session may be more experimental than another, on
the whole they are a critical tool to train bench scientists in
reproducible data analysis practices. We have provided some examples of
successful code club topics in Table 1 and summarized the results of our
own experimentation as Code Club presenters and participants into Ten
Rules. The first 3 rules apply to all participants at every Code Club,
while Rules 4 through 8 are targeted to Code Club presenters and the
last two for participants.

\subsection{Rule 1: Reciprocate
respect}\label{rule-1-reciprocate-respect}

It is critical that the presenter and participants respect each other
and that a designated individual (e.g.~the lab director) enforces a code
of conduct (see X for an example). Each member of the Code Club must
have the humility to acknowledge that they have more to learn about any
given topic. We have found that many problems are avoided when the
presenter takes charge of the session with a clear lesson plan,
thoughtful group creation, and constructive encouragement. Similarly,
participants foster a positive environment by remembering that the task
is not a competition, focusing on the presenter's goals, allowing their
partner to contribute, asking clarifying questions when appropriate, and
avoiding distractions (e.g.~email, social media). Learning to program is
challenging and too often results in a toxic environment of nitpicking
and nonconstructive criticism. All parties in a Code Club are
responsible for preventing this toxicity by demonstrating respect for
themselves and their colleagues.

\subsection{Rule 2: Let the material change
you}\label{rule-2-let-the-material-change-you}

Part of the humility required to participate in a Code Club is
acknowledging that your training is incomplete and that it is possible
for you to learn something new. For participants, assume that the
presenter has a plan and follow their presented approach. After the Code
Club, try to incorporate that material into new code or refactor old
code. By practicing the material in a different context you will learn
the material better. For presenters, incorporate suggested changes into
your code. Either party may identify concepts that they are unsure of,
presenting opportunities for further conversation and learning.

\subsection{Rule 3: Experiment!}\label{rule-3-experiment}

The selection of content and structure for each Code Club is democratic
and distributed. If someone thinks a technique is worth learning and
wants to teach it, they have that power. If they want to experiment with
a different format, they are free to try it out. Members of the Code
Club need to feel like they have the power to shape the direction of the
group. If members are following these Rules, they will naturally reflect
on the skills and interests of the other members in the group. For
example, there is always turnover in a research group, making it
important to revisit basic concepts to teach it to new people and
provide a refresher to others. The group should also feel free to
experiment with the format and incorporate group feedback by ending with
a debrief discussing the pros and cons of new formats.

\subsection{Rule 4: Set specific goals}\label{rule-4-set-specific-goals}

In our early Code Clubs we noticed that if the presenter did not clearly
state their goals for the session, it often led to frustration for both
the presenter and participants. If the presenter shared their own code,
did they want participants to focus on their coding style or did they
want help incorporating a new package into their workflow? Participants
will always notice or ask about code concepts that are not the focus of
the exercise; a presenter with a specific goal can bring tangential
conversations back to the planned task. Where possible, presenters
should create a simplified scenario (i.e.~minimal, reproducible example
{[}4{]}), which can be helpful in focusing the participants. The
presenter should verify ahead of time that the simplified example works
and behaves the way they expect. Beyond the content, clear goals for
participant activities will help both parties stay on task and avoid
frustration. For more advanced learners, the presenter can create
stretch goals or give an activity with multiple stopping points where
participants would feel successful (e.g.~commenting code, creating
function, implementing function, refactoring function). Accomplishing
specific goals is more likely to result in a positive outcome for
presenters and participants.

\subsection{Rule 5: Keep it simple}\label{rule-5-keep-it-simple}

Our Code Club needs to fit within an hour time slot. When considering
their Code Club activity, the presenter should plan for an introduction
and brief instruction, time for participants to engage the material, and
time for everyone to report back within that hour. A typical schedule
for Code Club is 10 minutes of introduction and instruction, 30 minutes
of paired programming, 5 minutes to get groups to wrap up, and 10
minutes to report back to the group. Remember that learners may need up
to three times as long to complete a task that is simple for the
presenter, so Code Club is best kept simple. We once had a presenter try
to teach basic Julia syntax, but time was up by the time participants
installed the interpreter. Some tips to help keep it simple are to limit
the presented code chunks to less than 50 or, conversely, consider the
number of lines that might be required to accomplish a solution.

\subsection{Rule 6: Give participants time to
prepare}\label{rule-6-give-participants-time-to-prepare}

Similar to a Journal Club, the presenter should give participants a few
days (ideally a week) to prepare for the Code Club. Considering the
compressed schedule described in Rule 5, asking participants to download
materials beforehand is helpful to ensure a quick start. The presenter
should provide the participants with instructions on how to install
dependencies, download data, and get the initial code. This might also
uncover weak points in the presenter's plan and enable them to ensure
that the materials work as intended before the Code Club. We have found
that using GitHub repositories for each Code Club can help make
information, scripts, and data easily available to participants. Perhaps
a first Code Club could be using git and GitHub to engage in
collaborative coding. A lower barrier entry point is posting their code,
data, and information in a lab meeting-dedicated Slack channel or
sharing via email. Whatever method is used, the presenter should be sure
to communicate the topic and necessary materials with the participants
ahead of time.

\subsection{Rule 7: Don't give participants busy
work}\label{rule-7-dont-give-participants-busy-work}

Participants want to learn topics that will either be useful to them or
help their colleague (i.e., the presenter). Presenters should do their
best to satisfy those motivations, whether it is through the relevancy
of the concept or the data. It does not make sense to present a Code
Club on downloading stock market data if it is not useful or interesting
to the group. Similarly, participants should not be tasked with
improving the presenter's code if the presenter has no intention of
incorporating the suggestions. A list of packages or tasks that group
members are interested in could help an uninspired presenter choose a
topic that will make a rewarding Code Club.

\subsection{Rule 8: Include all levels of
participants}\label{rule-8-include-all-levels-of-participants}

As suggested by Rule 7, a significant challenge to presenting at Code
Club is selecting topics and activities that appeal to a critical mass
of the participants. This is particularly difficult if participants have
a wide range of coding experience, such as follows a turnover in group
membership. In these circumstances it can be useful to cover commonly
used tools but that might result in disinterest of more experienced
participants. We have identified several strategies to overcome the
challenges presented by participants at varying skill levels. Instead of
letting participants form their own pairs, the presenter can select
pairs of participants with either similar or differing skill levels,
depending on their goals. Partnerships between those with similar skill
levels requires the presenter to design appropriate activites each skill
level. We have found that commenting code is a good skill for beginners
since it forces them to dissect and understand a code chunk
line-by-line. An advantage of partnerships with disparate skill levels
is assigning a single task, which provides the presenter with a diverse
range of methods that achieve the same result. This approach to pairing
also helps to graft new members into the group that have emerging
programming skills. Regardless of how partners are selected, consider
asking the pairs to identify a navigator and a driver {[}5{]}. The
driver types at the computer while the navigator tells them what to
type, thus ensuring participation of both partners. Midway through the
activity, the presenter can have the partners switch roles.
Intentionally forming pairs can also engineer group interactions by
breaking up cliques, avoiding potentially disruptive partnerships, or
pairing reliable role models with new group members.

\subsection{Rule 9: Prepare in advance to maximize
participation}\label{rule-9-prepare-in-advance-to-maximize-participation}

It is not possible to fully participate in a Journal Club discussion
about a paper that the participant has not read. In that context, coming
to Code Club without having installed a dependency is similar to asking
a simple question about the Journal Club paper. Both instances show a
lack of respect and preparation. Just as a presenter must follow Rule 6
to provide materials ahead of time, participants must review the code in
advance, download the data sets, install the necessary packages, and
perhaps read up on the topic. If the Code Club is based on a paper or
chapters in a data science book (e.g., {[}6--8{]}) the participants
should read them before the session and consider how they might
incorporate the concepts into their own work.

\subsection{Rule 10: Participate}\label{rule-10-participate}

An essential ingredient of any Code Club is active participation from
all parties. Having an open laptop on the table and permission to use it
can feel like an invitation to get distracted by other work, emails, and
browsing the internet; fight that urge and focus on the presenter's
goals. Be respectful and allow your partner to contribute. Speak up for
yourself and force your partner to let you contribute. If the material
seems too advanced for you, it can be frustrating, and tempting to
mentally check out. Fortuntately, programming languages like R and
Python are generally expressive, which should allow you to engage with
the logic, even if the syntax is too advanced. Frankly, understanding
the logic of when to use one modeling approach over another is more
important than how to use it. If you understand the ``why'', the ``how''
will quickly follow. More experienced participants should aim to
communicate feedback and coding suggestions at a level that all
participants can understand and engage in. Regardless of skill levels,
your partner and the presenter put themselves at risk by revealing what
they do or do not know. Encourage them, and show your gratitude for
helping you learn something new, by fully participating in each Code
Club.

\subsection{Conclusion}\label{conclusion}

The most important rules are the first and last. Members of the Code
Club need to feel comfortable with other group members and sufficiently
empowered to try something new or ask for help. Aside from growing in
our programming skills, we have noticed two other benefits that help
create a positive work culture. First, we have intentionally interviewed
postdoc candidates on the days we hold Code Club. We make it clear that
they are not being assessed on their coding skills, but instead use it
as an opportunity to see how the candidate interacts with other members
of the research group in a relatively low stakes environment. At the
same time, the candidate can learn about the culture of the research
group through active participation. Second, members of other research
groups that feel isolated in growing their skills have integrated
themselves into our Code Club. This speaks to both the broader need for
Code Club and the likelihood of success when expanded to include a
larger group of individuals with broader research interests. Ultimately,
Code Club has improved the overall data analysis skills, community, and
research success of our lab by empowering researchers to seek help from
their colleagues.

\subsection{Acknowledgements}\label{acknowledgements}

All co-authors have been participants in the Code Clubs run as part of
PDS's lab meetings. They have had a critical role in shaping the
evolution of the sessions. AKH led a lab discussion of Code Clubs to
identify the key concepts presented, which were supplemented by
contributions from NL. The order of co-authors was determined by\ldots{}

\newpage

\subsection{References}\label{references}

\hypertarget{refs}{}
\hypertarget{ref-Lonsdale2016}{}
1. Lonsdale A, Penington JS, Rice T, Walker M, Dashnow H. Ten simple
rules for a bioinformatics journal club. Lewitter F, editor. PLOS
Computational Biology. 2016;12: e1004526.
doi:\href{https://doi.org/10.1371/journal.pcbi.1004526}{10.1371/journal.pcbi.1004526}

\hypertarget{ref-Wilson2014}{}
2. Wilson G, Aruliah DA, Brown CT, Hong NPC, Davis M, Guy RT, et al.
Best practices for scientific computing. Eisen JA, editor. PLoS Biology.
2014;12: e1001745.
doi:\href{https://doi.org/10.1371/journal.pbio.1001745}{10.1371/journal.pbio.1001745}

\hypertarget{ref-Kothiyal2014}{}
3. Kothiyal A, Murthy S, Iyer S. Think-pair-share in a large CS1 class.
Proceedings of the 2014 conference on innovation \& technology in
computer science education - ITiCSE 14. ACM Press; 2014.
doi:\href{https://doi.org/10.1145/2591708.2591739}{10.1145/2591708.2591739}

\hypertarget{ref-StackOverflow_MREE}{}
4. Stack Overflow. How to create a minimal, reproducible example. 2020.
Available:
\url{https://stackoverflow.com/help/minimal-reproducible-example}

\hypertarget{ref-Bockeler_2020}{}
5. Böckeler B, Siessegger N. On pair programming. 2020. Available:
\url{https://martinfowler.com/articles/on-pair-programming.html}

\hypertarget{ref-Wickham_2016}{}
6. Wickham H, Grolemund G. O'Reilly Media; 2016. Available:
\url{https://r4ds.had.co.nz}

\hypertarget{ref-Schloss_2018}{}
7. Schloss PD. The riffomonas reproducible research tutorial series.
Journal of Open Source Education. 2018;1: 13.
doi:\href{https://doi.org/10.21105/jose.00013}{10.21105/jose.00013}

\hypertarget{ref-Noble2009}{}
8. Noble WS. A quick guide to organizing computational biology projects.
Lewitter F, editor. PLoS Computational Biology. 2009;5: e1000424.
doi:\href{https://doi.org/10.1371/journal.pcbi.1000424}{10.1371/journal.pcbi.1000424}

\newpage

\textbf{Table 1. Examples of successful Code Club topics.}

\begin{longtable}[]{@{}ll@{}}
\toprule
\begin{minipage}[b]{0.24\columnwidth}\raggedright\strut
\textbf{Title}\strut
\end{minipage} & \begin{minipage}[b]{0.61\columnwidth}\raggedright\strut
\textbf{Description}\strut
\end{minipage}\tabularnewline
\midrule
\endhead
\begin{minipage}[t]{0.24\columnwidth}\raggedright\strut
base vs.~ggplot\strut
\end{minipage} & \begin{minipage}[t]{0.61\columnwidth}\raggedright\strut
Given input data and a figure, recreate the figure using R's base
graphics or ggplot syntax\strut
\end{minipage}\tabularnewline
\begin{minipage}[t]{0.24\columnwidth}\raggedright\strut
Snakemake\strut
\end{minipage} & \begin{minipage}[t]{0.61\columnwidth}\raggedright\strut
Given a bash script that contains an analysis pipeline, convert it to a
series of recipes (can also be done with GNU Make)\strut
\end{minipage}\tabularnewline
\begin{minipage}[t]{0.24\columnwidth}\raggedright\strut
DRYing code\strut
\end{minipage} & \begin{minipage}[t]{0.61\columnwidth}\raggedright\strut
Given script with repeated code, create functions to remove
repetition\strut
\end{minipage}\tabularnewline
\begin{minipage}[t]{0.24\columnwidth}\raggedright\strut
mothur and Vegan\strut
\end{minipage} & \begin{minipage}[t]{0.61\columnwidth}\raggedright\strut
Given a pairwise community dissimilarity matrix, compare communities
using the \texttt{adonis} function in the Vegan R package\strut
\end{minipage}\tabularnewline
\begin{minipage}[t]{0.24\columnwidth}\raggedright\strut
tidy data\strut
\end{minipage} & \begin{minipage}[t]{0.61\columnwidth}\raggedright\strut
Given a wide-formatted data table, convert it to a long, tidy-formatted
data table using tools from R's tidyverse\strut
\end{minipage}\tabularnewline
\begin{minipage}[t]{0.24\columnwidth}\raggedright\strut
GitFlow\strut
\end{minipage} & \begin{minipage}[t]{0.61\columnwidth}\raggedright\strut
Participants file and claim an issue to add their name to a README file
in a GitHub-hosted repository and file a pull request to complete the
issue\strut
\end{minipage}\tabularnewline
\begin{minipage}[t]{0.24\columnwidth}\raggedright\strut
Integrating R and Google docs\strut
\end{minipage} & \begin{minipage}[t]{0.61\columnwidth}\raggedright\strut
Scrape a Google docs workbook and clean the data to identify previous
Code Club presenters\strut
\end{minipage}\tabularnewline
\bottomrule
\end{longtable}

\end{document}
